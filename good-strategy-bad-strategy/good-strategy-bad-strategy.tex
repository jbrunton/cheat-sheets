\documentclass{article}
\usepackage[utf8]{inputenc}
\usepackage{multicol}
\usepackage{calc}
\usepackage{ifthen}
\usepackage[portrait]{geometry}
\usepackage{amsmath,amsthm,amsfonts,amssymb}
\usepackage{color,graphicx,overpic}
\usepackage{hyperref}
\usepackage{tabularx}
\usepackage{graphicx}

\graphicspath{ {images/} }

\newenvironment{nosepitemize}
{ \begin{itemize}
    \setlength{\itemsep}{0pt}
    \setlength{\parskip}{0pt}
    \setlength{\parsep}{0pt}     }
{ \end{itemize}                  }

\newenvironment{nosepenumerate}
{ \begin{enumerate}
    \setlength{\itemsep}{0pt}
    \setlength{\parskip}{0pt}
    \setlength{\parsep}{0pt}     }
{ \end{enumerate}                  }

% Turn off header and footer
\pagestyle{empty}

% Don't print section numbers
\setcounter{secnumdepth}{0}

% This sets page margins to .5 inch if using letter paper, and to 1cm
% if using A4 paper. (This probably isn't strictly necessary.)
% If using another size paper, use default 1cm margins.
\ifthenelse{\lengthtest { \paperwidth = 11in}}
    { \geometry{top=.5in,left=.5in,right=.5in,bottom=.5in} }
    {\ifthenelse{ \lengthtest{ \paperwidth = 297mm}}
        {\geometry{top=1cm,left=1cm,right=1cm,bottom=1cm} }
        {\geometry{top=1cm,left=1cm,right=1cm,bottom=1cm} }
    }

\begin{document}

\begin{center}
     \section{Good Stategy/Bad Strategy Cheat Sheet}
\end{center}

\begin{multicols}{2}

\noindent
The core of strategy work is discovering the \textbf{critical factors in a situation} and designing a way of \textbf{coordinating and focusing actions} to deal with those factors. A good strategy honestly acknowledges the challenges being faced and provides an approach to overcoming them, and should \textbf{create strength through its coherence} and through shifts in viewpoint.

A \textbf{strategy is a hypothesis} and its implementation an experiment. Actions yield information about what does and does not work, and a good leader will adjust their strategy accordingly.


\end{multicols}

\begin{center}
\section{Bad Strategy}
\end{center}

\begin{multicols}{2}

\noindent
There are four hallmarks of bad strategy:

\begin{nosepenumerate}
    \item \textbf{Fluff} is gibberish masquerading as strategic concepts or arguments.
    \item A \textbf{failure to face the challenge}. Bad strategy fails to recognize the challenge. When you cannot define the challenge, you cannot evaluate a strategy or improve it.
    \item \textbf{Mistaking goals for strategy}. Many bad strategies are just statements of desire rather than plans for overcoming obstacles.
    \item \textbf{Bad strategic objectives} which fail to address critical issues or are impracticable.
\end{nosepenumerate}

\columnbreak

\noindent
Bad strategy is common for the following reasons:

\begin{nosepenumerate}
    \item \textbf{Unwillingness to choose}. Good strategy involves focus and choice. Without this, a strategy is weak and amorphous.
    \item \textbf{Template-style strategy}. The popularity of ``vision-led leadership" and ``mission vision strategy" templates lets leaders and consultants dress up the obvious as strategy without having to analyze real challenges. 
    \item \textbf{New Thought} has emphasised positive thinking over critical thinking.
\end{nosepenumerate}

\end{multicols}

\begin{center}
\section{The Kernel of Good Strategy}
\end{center}

\begin{multicols}{2}

\noindent
The \textbf{kernel} of a strategy contains three elements. A good strategy may contain more than this kernel, but if it is absent then there is a problem. The kernel consists of:

\begin{nosepenumerate}
    \item A \textbf{diagnosis} that defines or explains the nature of the challenge, simplifying the complex reality by identifying the critical aspects of a situation.
    \item A \textbf{guiding policy} for dealing with the challenge and obstacles outlined in the diagnosis. 
    \item A set of \textbf{coherent actions} that are designed to carry out the guiding policy.
\end{nosepenumerate}

\subsection{Diagnosis}

A good deal of strategy work is trying to figure out what is going on. An insightful diagnosis can bring \textbf{new perspectives} to bear. It suggests a domain of action with \textbf{leverage} over the obstacles. At a minimum, it \textbf{classifies the situation}, opening access to knowledge about how analogous situations were handled in the past. Making the diagnosis explicit also allows the remainder of the strategy to be revised as circumstances change.

\subsection{Guiding Policy}

A guiding policy creates advantage by \textbf{anticipating} the actions and reactions of others, by \textbf{reducing the complexity and ambiguity} in the situation, by exploiting the \textbf{leverage} inherent in concentrating effort on a pivotal or decisive aspect of the situation, and by creating policies and actions that are \textbf{coherent}, each building on the other rather than canceling one another out.

\subsection{Coherent Action}

Strategy is about action. The strategy does not need to name all the actions which can be taken and events unfold, but it must provide sufficient clarity for \textbf{consistent and coordinated} actions to be taken to address the specific challenge in the diagnosis. This is the most basic source of leverage or advantage available in strategy.

This coherence must often be imposed on a system with decentralized decision-making, which is a costly undertaking. Thus, coordinated policies should be applied only when the gains are large, imposing only the essential amount of coordination.

\end{multicols}

\begin{center}
\section{Sources of Power}
\end{center}

\begin{multicols}{2}

\noindent
A good strategy works by harnessing power and applying it where it will have the greatest effect.

\subsection{Leverage}
In general, strategic leverage arises from a mixture of \textbf{anticipation}, insight into what is most \textbf{pivotal} or critical in a situation, and from a \textbf{concentrated} application of effort.

The most critical anticipations are about the behavior of others, especially rivals. Most strategic anticipation draws on the predictable ``downstream" results of events that have already happened, from trends already at work, from predictable economic or social dynamics, or from the routines other agents follow that make aspects of their behavior predictable.

A \textbf{pivot point} magnifies the effect of effort. It is a place where a small adjustment can unleash larger pent-up forces.

A \textbf{threshold effect} exists when there is a critical level of effort necessary to affect the system.

\subsection{Proximate Objectives}

A proximate objective is close enough at hand to be \textbf{feasible}. A proximate objective names a target that the organization can reasonably be expected to hit, even overwhelm.

A good leader manages complexity and ambiguity, passing onto the organization a simpler problem which is \textbf{solvable}. A proximate objective may be challenging but it should be solvable.

The more uncertain and dynamic a situation, the poorer your foresight will be, and thus the more proximate a strategic objective should be. A strategy in the face of uncertainty should take a strong position and \textbf{create options}, rather than attempt to look far ahead.

\subsection{Chain-Link Systems}

A system has a \textbf{chain-link} logic when its performance is limited by its weakest sub-unit, or link, so cannot be made stronger by strengthening the other links. When each link is managed separately, such a system can get stuck in a low-effectiveness state because there is no incentive to invest resources into improving any of the other links.

Improving a chain-link system requires identification of the \textbf{limiting factors}, and may require centralized direction to address those factors in turn before gains are seen.

\subsection{Design}

A strategy should be viewed as a \textbf{design}, rather than a plan or a choice, because of the importance of mutual adjustment of the resources and actions deployed. In design problems, where various elements must be arranged, adjusted, and coordinated, there can be sharply peaked gains to getting combinations right and sharp costs to getting them wrong.

A design-type strategy is an adroit configuration of resources and actions that yields an advantage in a challenging situation. Given a set bundle of resources, the greater the competitive challenge, the greater the need for the clever, tight integration of resources and actions. Given a set level of challenge, higher-quality resources lessen the need for the tight integration of resources and actions.

\subsection{Focus}

A good strategy is focused in two ways: through \textbf{coordinated policies} that produce power through their interacting and overlapping effects, and through the application of that power to the \textbf{right target}.

\subsection{Growth}

It has become an article of almost unquestioned faith that growth creates value in itself, justifying mergers and acquisitions. However, healthy growth is not engineered. It is the outcome of growing demand for special capabilities, or of a firm having superior products and skills. It is the reward for successful innovation, cleverness, efficiency, and creativity.

\subsection{Advantage}
Advantage is rooted in differences–in the asymmetries among rivals. It is the leader’s job to identify which asymmetries are critical—which can be turned into important advantages.

Increasing value requires a strategy for progress on at least one of four different fronts:

\begin{nosepitemize}
    \item Deepening advantages.
    \item Broadening the extent of advantages.
    \item Creating higher demand for advantaged products or services.
    \item Strengthening the isolating mechanisms that block easy replication and imitation by competitor.
\end{nosepitemize}


\subsection{Dynamics}

One source of comparative advantage is exogenous ``waves of change". There are five guideposts which may indicate such change.

The first guidepost demarks an industry transition induced by \textbf{escalating fixed costs}. The second calls out a transition created by \textbf{deregulation}. The third highlights \textbf{predictable biases in forecasting}. A fourth marks the need to properly assess \textbf{incumbent response to change}. And the fifth guidepost is the concept of an \textbf{attractor state}–a sense of direction for the industry based on overall efficiency.

\subsection{Inertia and Entropy}

Similar to the concept of entropy in thermodynamics, weakly managed organizations tend to become less organized and focused. Entropy makes it necessary for leaders to constantly work on maintaining an organization’s purpose, form, and methods even if there are no changes in strategy or competition. Inertia and entropy are important to understand because:

\begin{nosepitemize}
    \item Successful strategies often owe a great deal to the inertia and inefficiency of rivals.
    \item An organization’s greatest challenge may not be external threats or opportunities, but instead the effects of entropy and inertia.
\end{nosepitemize}

Organizational inertia generally falls into one of three categories:

\begin{nosepenumerate}
    \item \textbf{The inertia of routine}. An organization’s standard routines and methods act to preserve old ways of categorizing and processing information.
    \item \textbf{Cultural inertia}, from the elements of social behavior and meaning that are stable and strongly resist change.
    \item \textbf{Inertia by proxy}. A lack of response is not always an indication of sticky routines or a frozen culture. A business may choose to not respond to change or attack because responding would undermine still-valuable streams of profit. Those streams of profit persist because of their customers’ inertia—a form of inertia by proxy. Inertia by proxy disappears when the organization decides that adapting to changed circumstances is more important than hanging on to old profit streams.
\end{nosepenumerate}

\end{multicols}

\end{document}
